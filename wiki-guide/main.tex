\documentclass[a4paper, 11pt, twoside]{book}
\usepackage{sty/packages}

\begin{document}
\pagecolor{pagecolor}
\frontmatter

\tableofcontents
\newpage

\import{other/copyright.tex}
\pagecolor{pagecolor}


\section{Introduction}
This document serves as a comprehensive guide for creating and managing your project Wiki. It is organized into various sections that cover specific aspects of project management, collaboration, design, documentation, and technical implementation. \\
Throughout this guide, I will assume a successful activation of the team Wiki but will walk the reader through the fundamental aspects of setting up the Wiki, as well as providing a general overview of project documentation. I will emphasize key elements that are essential for the Wiki team, such as effective documentation, clarity, and organizational strategies. The coding section will specifically focus on React applications, explaining the basics and offering guidance on best practices and methods for creating a dynamic and user-friendly interface. \\ \newline
At the end of the guide, there will be an index list to help you easily find specific topics and navigate through the content more efficiently. \\
This guide also serves as documentation to the associated template wiki, which teams are free to use as a starting point. \\ \newline
It is a first version and I will be working on improving it, so any feedback or ideas can be sent to liliana.sanfilippo@uni-bielefeld.de or raised as issues on GitHub. \\
Documentation for other tools or packages by myself is not included, but wil be linked. \\ \newline
While updating the guide, some already finished sections will be re-added and the following aspects will be added:
\begin{itemize}
    \item An in-depth explanation on how to implement a citation React component
    \item Setting Global Variables in React
    \item Creating HTML Code with Python
    \item Wiki Thaw
    \item Utilizing Bootstrap
    \item FAQ
    \item Merge Conflicts
    \item Utilizing ChatGPT
    \item Where can I get help?
    \item Definitions of terms such as CSS path, CSS selector
    \item Errors and Troubleshooting
    \item html link targets
    \item Further Collaborations and Considerations with other subteams (e.g. Sponsoring and logos)
    \item Scrolling animations
    % https://stackoverflow.com/questions/16230959/what-does-top-in-the-hyperlink-target-do
\end{itemize}
As well as additional images outlining the instructions and examples given.
\newpage

\mainmatter
\pagecolor{pagecolor}
% Getting started is done
\chapter{Getting started} \label{sec:started}
%\subfile{sections/started}
%\stand{June 2025}
\newpage
% Design done (can be expanded)
\chapter{Design} \label{sec:design}
%\stand{June 2025}
%\subfile{sections/design}
\newpage

\backmatter
\pagecolor{pagecolor}
\section*{Attributions}
\import{other/attributions.tex}

\end{document}