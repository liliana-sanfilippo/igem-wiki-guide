\documentclass[a4paper, 11pt, twoside]{book}
\usepackage{import}
\import{sty/packages.sty}


\begin{document}
\pagecolor{pagecolor}
\frontmatter

\tableofcontents
\newpage

\import{other/copyright.tex}
\pagecolor{pagecolor}


\section{Introduction}
This document serves as a comprehensive guide for creating and managing your project Wiki. It is organized into various sections that cover specific aspects of project management, collaboration, design, documentation, and technical implementation. \\
It is a first version and I will be working on improving it, so any feedback or ideas can be sent to liliana.sanfilippo@uni-bielefeld.de \\
Your contributions can help us enhance the resource for all teams involved in the iGEM project. \\ \newline
Throughout this guide, we assume a successful activation of the team Wiki but will walk the reader through the fundamental aspects of setting up the Wiki, as well as providing a general overview of project documentation. We'll emphasize key elements that are essential for the Wiki team, such as effective documentation, clarity, and organizational strategies. The coding section will specifically focus on React applications, explaining the basics and offering guidance on best practices and methods for creating a dynamic and user-friendly interface. \\ \newline
Additionally, the document includes an appendix containing full examples that are briefly mentioned in the sections. This will provide a deeper understanding of the concepts discussed. At the end of the guide, there is an index list to help you easily find specific topics and navigate through the content more efficiently
While updating the guide for our next team, the following aspects will be added:
\begin{itemize}
    \item An in-depth explanation on how to implement a citation React component
    \item Setting Global Variables in React
    \item Creating HTML Code with Python
    \item Wiki Thaw
    \item Utilizing Bootstrap
    \item FAQ
    \item Merge Conflicts
    \item Utilizing ChatGPT
    \item Where can I get help?
    \item Definitions of terms such as CSS path, CSS selector
    \item Errors and Troubleshooting
    \item html link targets
    \item Further Collaborations and Considerations with other subteams (e.g. Sponsoring and logos)
    \item Scrolling animations
    % https://stackoverflow.com/questions/16230959/what-does-top-in-the-hyperlink-target-do
\end{itemize}
As well as additional images outlining the instructions and examples given.
\newpage

\mainmatter
\pagecolor{pagecolor}
% Getting started is done
\chapter{Getting started} \label{sec:started}
%\stand{June 2025}


\backmatter
\pagecolor{pagecolor}
\section*{Attributions}
\import{other/attributions.tex}

\end{document}