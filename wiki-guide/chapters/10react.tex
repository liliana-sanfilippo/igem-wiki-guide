%! Author = lili
%! Date = 04.09.25


\section{Getting started with your React project} \label{sec:react-start}
\paragraph{Updating Package Managers:}  If you already have \texttt{npm} or \texttt{yarn} installed, consider
updating them to the latest version to avoid compatibility issues.
You do not need both package managers, one should be sufficient.
If you are unsure, read about the
\href{https://www.geeksforgeeks.org/node-js/difference-between-npm-and-yarn/}{differences} between the managers to
make a decision. \\
Otherwise \textbf{Install Package Mangers:}  \index{Package Managers} \index{npm} \index{Node.js} \index{yarn}
\begin{itemize}
    \item \textbf{node}: Make sure you have Node.js installed.
    You can check by running \texttt{node --version}.
    If it’s not installed, download it from the \href{https://nodejs.org/}{Node.js website}.
    \item \textbf{npm:} It should come bundled with Node.js.
    \item \textbf{yarn}: If you don't already have yarn installed, you can usually do so by running the following
    command: \begin{verbatim}
    npm install --global yarn
    \end{verbatim}
\end{itemize}

\paragraph{Install Project Dependencies:} Navigate to your project directory in the terminal and run:
\begin{verbatim}
    yarn install
\end{verbatim}
or
\begin{verbatim}
    npm install
\end{verbatim}

\paragraph{Ensure Compatible Node.js Version:} Make sure your Node.js version is either \texttt{\^\ 8.18.0} or \texttt{>=20.0.0} to avoid compatibility issues, with \texttt{18.20.0} being the recommended version in 2024. \index{Node.js}
