%! Author = lili
%! Date = 6/30/25

Creating well-structured documentation\index{Documentation} is essential for effective communication and information retention.
Here are some general considerations for documentation\index{Documentation}, including various methods to create PDFs and layout decisions for protocols and notes.
\section{General considerations} \label{sec:general-considerations}
\subsection*{Purpose and Audience}
Clearly define the purpose of the document and be aware your target audience.
Understanding the audience's knowledge level will guide your writing style and content depth.
Of course judges will most likely come from a biotech background, but they can still be unfamiliar with niche topics.
A different example where your documentation and documents need audience considerations would be flyers or videos for educational purposes, especially if you want to choose education as a special prize.

\subsection*{Clarity and Consistency}
Use clear, concise language to convey information effectively.
Maintain consistency in terminology, formatting, and style throughout the document to enhance readability and comprehension.
There is no special prize for using the most complicated words or coming up with the most intricate way to describe simple procedures.

\subsection*{Sustainability}
Decide on styles and approaches early on and think about using English from the get-go to avoid having to translate and rework your notes and protocols later on.
Remember that you need to credit translators and potentially translating tools in the attributions. \index{attributions}

\subsection*{Structure and Organization}
Use a logical structure with a clear hierarchy, even if it seems basic or unnecessary to you.
\begin{itemize}
    \item \textbf{Title Page:} Include the document title, author(s), date, and possibly your logo.
    Remember, the document is presented on your wiki, but it is possible it will be distributed in a different context, and therefore you should always add all necessary information to each document.
    \item \textbf{Table of Contents:} Do not forget to provide a navigation aid for longer documents.
     If possible, create a table of content with hyperlinks.
    \item \textbf{Structure} Divide content into clear sections with appropriate headings.
    Possibly use a glossary for intricate documents.
    \item \textbf{Conclusion:} Summarize key points to ensure the reader understands the main points and offer recommendations if applicable.
\end{itemize}


\subsection*{Choosing a Format for Documentation\index{Documentation!Format}}
\paragraph{Word Processing Software (e.g., Microsoft Word, Google Docs):} These tools allow for straightforward document creation with various formatting options.
You can export your documents as PDFs easily, ensuring compatibility across different devices and allowing collaborative writing.
\paragraph{LaTeX:} Ideal for more complex documents, LaTeX offers precise control over formatting, making it perfect for scientific papers and technical documentation\index{Documentation}.
It is particularly beneficial for documents with mathematical equations, citations, and references.
Services such as Overleaf allow collaborative writing. \\  \newline
It is possible to create PDF-masks with design tools such as Canva, if you want to add further styles or branding to your documents.\\ \newline
Even if you are unsure what to use yet, a digital text file of any sort is always easier to edit and transfer than handwritten notes!



\section{Team Meetings} \label{sec:team-meetings} \index{Documentation!Meetings}
It is advisable to have meeting protocols from the beginning.
Depending on your team structure, experience and regarding time management, you should consider:
\begin{itemize}
    \item \textbf{Preparing agendas}: Starting meetings with a pre-prepared agenda outlining the topics for discussion.
    \item This helps keep the meeting focused and provides a structure for your notes.
    \item \textbf{Minute taker}: Assign the task of note keeping to a specific person every meeting.
    \item \textbf{Tracking decisions}: Document any decisions or agreements reached during the meeting to provide a clear record of team choices.
    \item \textbf{File Storage and Accessibility:} The meeting notes should be accessible to the whole team afterward and multiple versions should be most urgently avoided.
    \item \textbf{Summaries}: Some teams may find it helpful to create meeting summaries after the fact.
    \item Though this can also be a time drain.
\end{itemize}
\section{Lab} \label{sec:lab} \index{Documentation!Lab}
Well-maintained lab books are essential for tracking progress, reproducing experiments, and ensuring accountability.
The following aspects seem basic and natural which unfortunately puts them at risk of being forgotten for this exact reason.  \index{lab book}
\begin{itemize}
    \item \textbf{Digitalize}: Most projects get busier towards the end, and it is unlikely you will have the time to digitalize your notes later on.
    It is best to keep it digitally from the beginning, in whatever exact form.
    Even a \texttt{.txt} file is better than handwritten notes, because the text can simply be copied to the wiki in
    the worst case.
    \item \textbf{Backup}: Regularly back up entries to prevent data loss.
    \item \textbf{Raw Data}: Always include raw data alongside processed or analyzed data and save the data separately, too.
\end{itemize}

\noindent If you want to use a lab book software, please check beforehand if it allows you to export your (raw) data and notes in a useful way. \index{lab book!software} Usually, useful ways are easily processable data files such as \texttt{.cvs} or \texttt{.json}.  \\ \newline

\subsection*{Protocol design examples}
\begin{itemize}
    \item \href{https://static.igem.wiki/teams/5087/pics/protocols/rna-purification.pdf}{JU Krakow 2024}
\end{itemize}

\section{Integrated Human Practices} \label{sec:ihp} \index{Documentation!IHP} \index{Integrated Human Practices}
Documenting Integrated Human Practices involves tracking contacts, conversations, permissions, and the use of any media, ensuring compliance with ethical and legal standards. \newline
Both to avoid unnecessary work and to maintain a professional demeanor towards your stakeholders.
\begin{itemize}
    \item \textbf{Tracking Contacts and Interactions}:
    \begin{itemize}
        \item Maintain a list of \textbf{Contact Information} of the individuals and institutions you both want to contact and already contacted.
        Include details such as \textit{role, affiliation, date of first contact,} etc.
        \item Keep track of \textbf{Who Contacted Whom} to avoid contacting the same person multiple times.
        This helps in maintaining accountability and tracking follow-ups.
    \end{itemize}
    \item \textbf{Consent Management} \index{Consent}
    \begin{itemize}
        \item Be sure to retain \textbf{Informed Consent for Media Usage} from the people you interviewed or otherwise created media content with.
        You should mention how the ``freezing'' of iGEM Wikis works and that it will not be possible to change or remove information after the project ends.
        \item Be mindful to receive \textbf{Feedback for Quotes and Transcripts\index{Transcripts}} of interviews. \index{Interview} Especially if you need to translate conversations to English, the stakeholders should get the chance to comment and, if necessary, correct your translations.
        \item Remember to ensure a tidy \textbf{Storage of Consent Information} that is accessible to the whole team.
    \end{itemize}
    \item \textbf{Recording Conversations and Outcomes }
    \begin{itemize}
        \item If possible, clear the \textbf{Type of Documentation} with the interviewee beforehand.
        \item Try to organize \textbf{High Quality Tools} to record and test them beforehand.
        \item Be aware of background noises and keep in mind that the videos could be useful for your project presentation video.
        \item Create \textbf{Meeting Summaries} including \textit{main topics}, \textit{key takeaways}, \textit{quotes} and \textit{to-dos}.
    \end{itemize}
    \item \textbf{Categorizing Stakeholders and Their Input}
    \begin{itemize} \index{Stakeholders}
    \item The \textbf{Categorization of Contacts} into categories such as \textit{Academia, Industry} or \textit{Community} should be started as soon as possible to streamline documentation\index{Documentation} and to facilitate references back to relevant discussions.
    \item Keep track of the \textbf{Implementation of Advice and Input} you received to be able to cross-reference from you Integrated Human Practice to other aspects of your project.
    \end{itemize}
    \item \textbf{Ethical Considerations}
    \begin{itemize}
        \item Ensure that no \textbf{Sensitive Information} is included in your published material.
        This can include \textit{confidential project data} or \textit{compromising information}.
        If necessary, censor details to ensure the \textbf{Privacy} of individuals such as patients, children or other vulnerable stakeholders.
        \item Maintain proper \textbf{Attribution} for both intellectual input and media usage.
    \end{itemize}
    \item \textbf{Maintaining Transparency}
    \begin{itemize}
        \item Maintain \textbf{Transparency in Reporting} by including clear and concise references to your interactions with stakeholders when writing Wiki texts.
        \item Be upfront about \textbf{Changes and Censorship} in your documentation\index{Documentation}.
    \end{itemize}
\end{itemize}
\section{MeetUps} \label{sec:meetups} \index{Documentation!MeetUps}
When documenting an iGEM meetup, it's important to capture not only the logistics but also the valuable exchanges, collaborations, and outcomes from the event.
\begin{itemize}
    \item \textbf{Photos and Videos}: Be sure to organize media documentation\index{Documentation} beforehand and take pictures of key moments.
    Not only for your team but to support the attending teams and allow them to concentrate on the event.
    \item \textbf{Expert Feedback\index{Expert Feedback}}: If experts or judges were present, document the feedback they provided to your team or others.
    Highlight how this feedback can influence your project moving forward.
    \item \textbf{Material Exchange}: If any teams shared protocols, tools, or resources during the meetup, keep track of what was exchanged and which team shared it.
    \item \textbf{Lessons Learned}: Capture insights or takeaways from the event that could influence your project, approach, or team dynamics moving forward.
    \item \textbf{Online Resources}: If presentations, slides, or meeting recordings were shared, document how these materials can be accessed later (e.g., through a shared drive or link).
    \item \textbf{Meetup Agenda}: Include the event’s agenda or a summary of the planned sessions, presentations, or workshops.
    Be sure to note changes.
    \item \textbf{Feedback}: If possible, gather feedback on-site at the end of the event to ensure high involvement and profit from fresh impressions.
\end{itemize}
