
    \section{Reading iGEMs resources} \label{sec:reading-igem-resources}
    To successfully build your iGEM team Wiki, it is essential to thoroughly review iGEM’s official Wiki Rules and Resources. 
    Pay close attention to size limitations, time restrictions, and functionality constraints to ensure compliance with the competition requirements. 
    Familiarize yourself with best practices and common pitfalls by examining previous Wikis, particularly those from past winning teams, as they can provide valuable insights into effective strategies and potential challenges. \newline
    Before you dive into coding, make sure to read iGEM’s \href{https://competition.igem.org/deliverables/team-Wiki}{official information on the team Wikis} and the \href{https://competition.igem.org/deliverables/gitlab-guide}{GitLab Guide}.
    These documents will help you understand the basic requirements, technical constraints, and available tools necessary for your Wiki's development. 
    Key resources to consult include the iGEM GitLab Guide and the Team Wiki Deliverable page, which offer an overview of important documents that will guide you throughout the process. \newline
    By following these steps, you will be better prepared to create a well-structured and compliant Wiki that effectively showcases your team’s project.


    \section{Deciding on a template \index{Template}} \label{sec:template-decision}
    When selecting a template for your iGEM team Wiki, it’s important to consider your team’s skill level to ensure a smooth development process:

    \begin{itemize}
        \item \textbf{Markdown \& Frozen Flask (Beginner)}: This simple template utilizes Markdown files to create a static website. 
        It's user-friendly, making it an ideal choice for teams just starting out.
        \item \textbf{HTML \& Frozen Flask (Intermediate)}: This template employs HTML files to generate a static website. 
        While it is slightly more complex than the Markdown option, it remains relatively straightforward to use, making it suitable for teams with some coding experience.
        \item \textbf{React (Advanced)}: This option is for teams with advanced skills in React, a complex framework that allows for the creation of dynamic and interactive Wikis. 
        Before choosing this template, ensure that your team is proficient in React, as we will not provide detailed guidance on its usage.
        It’s crucial to review the requirements and thoroughly test your Wiki before the Wiki Freeze to ensure everything functions as intended.
    \end{itemize}

    \section{Let everyone know the essentials \index{Essentials}} \label{sec:the-essentials}
    Everyone should be aware of the requirements and mistakes that could get you disqualified to avoid the responsibility being placed on one person.
    Important aspects are:
    \begin{itemize}
        \item GitLab Repository Link
        \item Licensing \index{Licensing}
        \item Standard URL Pages \index{Standard URL Pages}
        \item Content Hosting
        \item iframes Usage
        \item Source Code Submission
    \end{itemize}
    Do not rely on the knowledge of former teams only as the guidelines can change.


    \section{Distributing responsibility} \label{sec:distributing-responsibility}
    To effectively build your iGEM team Wiki, it’s essential to clearly define team roles and responsibilities.
    Consider assigning specific positions such as content creators, developers, and designers.
    Utilizing task management tools can help streamline the process and ensure everyone stays organized.
    Each team member should be aware of their specific tasks related to the Wiki, which may include:
    \begin{itemize}
        \item \textbf{Page Layouts and Design:} Creating visually appealing and user-friendly layouts.
        \item \textbf{Coding:} Implementing the necessary code for functionality.
        \item \textbf{Automation:} Streamlining repetitive tasks where possible.
        \item \textbf{Data Curation:} Organizing and managing data sets relevant to your project.
        \item \textbf{Media Management:} Uploading videos, taking and editing photographs (including considerations for lighting, cropping, and naming conventions).
        \item \textbf{Text Creation and Editing:} Writing, editing, and correcting text to ensure clarity and accuracy.
        \item \textbf{Task Management:} Overseeing progress and reminding team members of their responsibilities.
        \item \textbf{Mobile Optimization:} Ensuring the site is accessible on mobile devices.
        \item \textbf{Illustrations and Animations:} Creating visual elements to enhance the content.
        \item \textbf{Accessibility:} Implementing features that make the Wiki usable for all visitors.
        \item \textbf{Documentation:} Maintaining accurate records of processes and changes.
        \item \textbf{Development of Additional Tools:} Creating helpful scripts or tools, such as Python scripts, to automate tasks.
    \end{itemize}
    By clearly defining roles and responsibilities and utilizing task management tools, your team can effectively collaborate to create a comprehensive and engaging Wiki for the iGEM competition.
    It is recommended to distribute these tasks on multiple people. \newline
    Most tasks concerning the Wiki overlap with other subteams and should therefore be handled in cooperation with these.

%\section{Activating your Wiki}
